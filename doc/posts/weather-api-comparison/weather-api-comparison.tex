\documentclass[12pt]{article}

\usepackage[margin=1in]{geometry}
\usepackage{gensymb}
\usepackage{svg}
\usepackage{url}

\setsvg{svgpath = ../../figures/weather-api-comparison/}

\begin{document}
  \title{Comparing Weather APIs}
  \author{Jonathan Lee}
  \date{February 4, 2016}
  \maketitle

  \section{Introduction}
    The SARBayes project uses the International Search \& Rescue Incident
    Database (ISRID) \cite{isrid} to study and forecast lost person behavior.
    To augment the predictive power of the project's models, we can
    supplement sparsely populated fields in ISRID with other sources of data.
    For instance, given an incident's date and location, we can pull data from
    online application programming interfaces (APIs) to fill in missing values
    for weather conditions such as temperature and precipitation.

    We considered two historical weather data sources: the National Oceanic and
    Atmospheric Administration's (NOAA) Climate Data Online Web Service and
    Weather Services International's (WSI) Cleaned Observation API. This post
    presents how closely the measurements provided by these two services
    matched the values observed by search-and-rescue personnel on the ground
    and recorded in the database. On average, WSI had better coverage and was
    \(2 \degree C\), \(1 \degree C\), and \(3 km/h\) closer in maximum
    temperature, minimum temperature, and wind speed, respectively.

  \section{Methods}
    \subsection{Error Calculation}
      The features we compared are daily maximum temperature, daily minimum
      temperature, and wind speed. We found no quantitative measurements for
      snow or rain in ISRID, and excluded these features from the test. For
      each case with a valid date and latitude and longitude coordinates, we
      pulled ``experimental" values from each API. If both an observed value and
      an experimental value for a particular feature were available, we
      calculated the error of that API's feature as

      \[error_{feature} = experimental_{feature} - observed_{feature}\]

      That is, the error measures how far off the experimental value is and in
      which direction. A positive error indicates overestimation by the API,
      and a negative value indicates underestimation. The error shares the same
      units as its feature.

      It is worth noting that the observations made and recorded by searchers
      are by no means perfect. For instance, we found several hundred instances
      where the high temperature was exactly the same as the low temperature,
      which may be an entry error. However, given how many cases occur in areas
      where conditions may vary greatly across small distances, such as
      mountainous terrain, we assume the data collected by searchers near the
      exact site of the incident best describe the actual conditions the lost
      person experienced.

    \subsection{API Access Implementation}
      Readers willing to trust the validity of our data collection process are
      free to skip to the Results section.

      \subsubsection{NOAA}
        To access the Climate Data Online Web Service \cite{noaa}, users must
        send HyperText Transfer Protocol (HTTP) requests to a valid API
        endpoint using the \texttt{GET} method. Each request requires an access
        token, which is initially limited to 1000 requests per day and is
        available free of charge to anyone who registers an email address with
        NOAA. The API returns the requested data as plain text in compressed
        JavaScript Object Notation (JSON) form.

        Within the web service, measurements have associated data types, which
        are listed at the \texttt{datatypes} endpoint. We mapped the
        \texttt{TMAX} data type to the daily maximum temperature feature in
        ISRID, \texttt{TMIN} to daily minimum temperature, and \texttt{AWND} to
        wind speed. The API reports values for temperature in tenths of a
        degree Celsius and for wind speed in tenths of a meter per second.

        For each case, we calculated the bounds enclosing a square region with
        the initial planning point (IPP) at the square's center. We arbitrarily
        chose 40 km as the length of each side, and assumed weather conditions
        would be similar within the region. Then, we queried the
        \texttt{stations} endpoint with the bounds and date to obtain a set of
        relevant station identifiers.

        To obtain the actual measurements, we sent the date, set of station
        identifiers, and set of needed datatypes to the \texttt{data} endpoint.
        If the service reported more than one value per data type, we took a
        simple average. Each case consumed two requests, at most.

      \subsubsection{WSI}
        Users may access the Cleaned Observation API \cite{wsi} by sending
        \texttt{GET} requests with the required parameters and key to
        \url{http://cleanedobservations.wsi.com/CleanedObs.svc/GetObs}. We
        received a free key during a trial period to conduct this test.

        Unlike Climate Data Online, we merely neeeded to specify latitude and
        longitude coordinates to access historical data; the API manages data
        compilation internally. Also, measurements are offered for every hour.
        We obtained daily maximum and minimum temperatures by taking the
        extremes of a list of values for the \texttt{surfaceTemperatureCelsius}
        data type, the dry bulb temperature two meters above the surface. To
        obtain the average daily wind speed, we averaged that day's values for
        \texttt{windSpeedKph}, the unobstructed wind speed 10 meters up. The
        API reported temperatures in degrees Celsius and wind speeds in
        kilometers per hour.

  \section{Results}
    We found 3049 observed values for daily high temperature, 1824 for daily
    low temperature, and 1456 for wind speed. By taking the absolute value of
    each error value, we summarized the key statistics of each feature in the
    following tables.

    \renewcommand{\arraystretch}{1.25}

    \begin{table}[h]
      \centering
      \caption{NOAA API Error Statistics}
      \begin{tabular}{l c c c c}
        \hline
        Feature & Count & Coverage (\%) & Mean & Standard Deviation \\
        \hline
        Daily Maximum Temperature & 2952 & 96.8 & 7.4 \degree C & 7.1 \degree C
        \\
        Daily Minimum Temperature & 1750 & 95.9 & 8.3 \degree C & 7.5 \degree C
        \\
        Daily Average Wind Speed & 428 & 29.4 & 13.1 km/h & 13.5 km/h \\
        \hline
      \end{tabular}
    \end{table}

    \begin{table}[h]
      \centering
      \caption{WSI API Error Statistics}
      \begin{tabular}{l c c c c}
        \hline
        Feature & Count & Coverage (\%) & Mean & Standard Deviation \\
        \hline
        Daily Maximum Temperature & 3048 & 99.9 & 5.5 \degree C & 4.8 \degree C
        \\
        Daily Minimum Temperature & 1823 & 99.9 & 7.2 \degree C & 7.1 \degree C
        \\
        Daily Average Wind Speed & 1455 & 99.9 & 10.3 km/h & 11.9 km/h \\
        \hline
      \end{tabular}
    \end{table}

    For each feature, we also sorted the signed error values into 50
    evenly-sized bins and charted the frequency of the bins as histograms.
    The normal distributions for each list of error values are overlaid on the
    plots as dashed curves. Distributions tightly clustered around 0, no error,
    indicate better conformance to the observed values.

    \includesvg[width=\textwidth]{daily-max-temp-error}
    \includesvg[width=\textwidth]{daily-min-temp-error}

    Both APIs tend to overestimate values for both types of temperature. Daily
    maximum temperature estimates from both APIs tend to exhibit less variance
    than minimum temperature estimates; perhaps we can attribute this behavior
    to the aforementioned cases where the maximum temperature may have been
    carried over into the minimum temperature field. Interestingly, for both
    temperature features, NOAA has a noticable ``tail" of high-error cases on
    the right where many estimates were off by more than 20 degrees Celsius.

    \includesvg[width=\textwidth]{daily-avg-wind-speed-error}

    NOAA also has many overestimates for wind speed, which may be responsible
    for dragging the mean of the absolute error up.

  \section{Conclusion}
    For the purpose of populating ISRID, WSI is the better of the two services,
    offering nearly universal coverage, lower error, and greater ease-of-use.

    NOAA may have underperformed because the cases we sampled were not
    representative of the entire database. Searchers in different geographic
    regions may have different procedures and resources for collecting weather
    data, possibly excluding certain areas from the test where NOAA has more
    stations and performs particularly well. We may have also introduced error
    into NOAA's measurements by taking a simple average of values from multiple
    stations. Because few subjects actually reached the edges of the regions we
    sampled from, na\"{i}vely treating all stations equally may have harmed
    the API's accuracy. An average weighted on distance from the IPP can remedy
    this issue.

    \subsection{Future Work}
      Future work may include importing geospatial data and using the IPP and
      find location to populate fields such as elevation change and track
      offset. Also, survival models may benefit from measurements made closer
      to the time of the incident by using an hourly, rather than daily,
      resolution.

  \bibliography{weather-api-comparison}
  \bibliographystyle{unsrt}
\end{document}
